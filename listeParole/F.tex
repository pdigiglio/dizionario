
\newglossaryentry{florilegio}{% 10
    name        = {Florilegio},%
    description = {\switch{1}~\lett~Raccolta di brani tratti dall'opera di uno o più scrittori, antologia: \ex{florilegio carducciano}. \switch{2}~\eccl~Raccolta contenente preghiere e vite di santi. \switch{3}~Raccolta di brutture, modelli da non imitare: \ex{una prosa che è un florilegio di trivialità}}%
}

\newglossaryentry{fiele}{% 5
    name        = {Fiele},%
    description = {\switch{1}~Bile. \switch{2}~\fig~Amarezza, dispiacere. \switch{3}~Odio, rancore rabbioso e spesso celato}%
}

\newglossaryentry{falcidiare}{% 10
    name        = {Falcidiare},%
    description = {\fig~Dare una falciata, fare una falcidia, decimare, defalcare}%
}

\newglossaryentry{faretra}{% 7
    name        = {Faretra},% 2016-05-10
    description = {Astuccio atto a conservare frecce che gli arcieri portavano sospeso alla spalla destra per mezzo di una correggia}%
}

\newglossaryentry{forania}{% 7
    name        = {Foran\`ia},% 2016-05-21
	description = {Carica e ufficio di \emph{vicario foraneo} (parroco preposto ad uno dei distretti, comprendenti più parrocchie, in cui si può dividere una diocesi); distretto su cui si estende la sua giurisdizione, anche detto \emph{vicariato foraneo}}%
}

\newglossaryentry{facinoroso}{% 10
    name        = {Facinoroso},% 2016-06-30
    description = {Violento, ribelle, turbolento}%
}
