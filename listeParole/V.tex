
\newglossaryentry{vulgata}{% 7
    name        = {Vulgata},%
    description = {Versione latina della Bibbia dovuta in massima parte a S.~Girolamo, la sola dichiarata autentica e adottata dalla Chiesa romana}%
}

\newglossaryentry{vernacolo}{% 9
    name        = {Vernacolo},%
    description = {\switch{1}~Parlata caratteristica di una zona limitata. Si distingue da \emph{lingua} e \emph{dialetto}, rispetto al quale è più popolare e locale. \switch{2}~\agg~Nativo, paesano}%
}

\newglossaryentry{vegliardo}{% 9
    name        = {Vegliardo},%
    description = {Uomo d'età molto avanzata. Vecchio che ispira rispetto e venerazione}%
}

\newglossaryentry{volitivo}{% 8
    name        = {Volitivo},%
    description = {Che è dotato di grande forza di volontà: \ex{un uomo volitivo}. Anche di gesti o atteggiamenti che manifestano tale volontà}%
}

\newglossaryentry{vivandiere}{% 10
    name        = {Vivandiere},% 2016-05-07
    description = {\mil~In accampamenti, caserme \etc, chi vende(va) cibi e bevande ai soldati}%
}

\newglossaryentry{virago}{% 6
    name        = {Virago},% 2016-05-10
    description = {Donna d'animo e robustezza virile}%
}
