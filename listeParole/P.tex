
\newglossaryentry{pertugio}{% 8
    name        = {Pertugio},%
    description = {Buco, fessura. Per iperbole, apertura o passaggio eccessivamente angusto}%
}

\newglossaryentry{perspicuo}{% 9
    name        = {Perspicuo},%
    description = {\switch{1}~Che si lascia attraversare dalla luce, trasparente, diafano. \switch{2}~Di facile comprensione}%
}

\newglossaryentry{piccato}{% 7
    name        = {Piccato},%
    description = {Risentito, impermalito}%
}

\newglossaryentry{precipuo}{% 8
    name        = {Precipuo},%
    description = {Principale, essenziale, fondamentale, tipico, caratteristico}%
}

\newglossaryentry{periplo}{% 7
    name        = {Periplo},%
    description = {Circumnavigazione di un'isola o continente. \est~Viaggio circolare in aereo o altro mezzo, anche figurato}%
}

\newglossaryentry{puerperio}{% 9
    name        = {Puerperio},% 2016-05-10
    description = {\fisiol~il periodo di 6-8 settimane tra il parto e la ripresa della normale attività ovarica}%
}

\newglossaryentry{peculato}{% 8
    name        = {Peculato},% 2016-05-13
    description = {Reato commesso da un pubblico ufficiale o dall'incaricato di un pubblico servizio che volontariamente volge a beneficio proprio o altrui il denaro o altro bene mobile di cui è in possesso per ragioni del proprio ufficio}%
}

\newglossaryentry{prebenda}{% 8
    name        = {Prebenda},% 2016-05-18
    description = {\switch{1}~\eccl~Rendita assegnata ad un ecclesiastico in relazione al suo grado e al suo ufficio: \ex{prebende canonicali}. \switch{2}~\est~Guadagno facile e più o meno illecito. \switch{3}~\ant~Razione di biada o foraggio}%
}

\newglossaryentry{probatorio}{% 10
    name        = {Probatorio},% 2016-06-03
    description = {\giur~Attinente alle prove o che ha forza di prova: \ex{documento probatorio}. Durante il processo, \ex{istruzione probatoria} è la fase diretta al raccoglimento delle prove}%
}

\newglossaryentry{peculio}{% 7
    name        = {Peculio},% 2016-06-06
    description = {\switch{1}~Bestiame, gregge, armento. \switch{2}~Nel diritto romano, il piccolo patrimonio che il \emph{pater familias} soleva concedere al figlio, e talora ad un servo, perché ne avesse il godimento e l'amministrazione ma non la proprietà (analogamente nel diritto canonico: \ex{peculio dei clerici}). \switch{3}~Somma di denaro che uno tiene da parte o porta con sé}%
}

\newglossaryentry{pinzocchero}{% 11
    name        = {Pinzocchero},% 2016-06-06
    description = {(anche \emph{pinzochero} oppure \emph{pizzocchero}) --- \switch{1}~\ant~Nel medioevo, persona che, appartenendo come laico ad un ordine o congregazione religiosa, conduceva vita devota, di preghiera e carità. \switch{2}~\est~Persona che ostenta una religiosità puramente esteriore; bacchettone, bigotto}%
}

\newglossaryentry{pernicioso}{% 10
    name        = {Pernicioso},% 2016-06-13
    description = {Che provoca danni molto gravi: \ex{lo scontro tra fazioni avrebbe effetti perniciosi}}%
}

\newglossaryentry{pizzicagnolo}{% 12
    name        = {Pizzicagnolo},% 2016-06-29
    description = {(da \emph{pizzicare}, perché vende generi commestibili piccanti)~--- Negoziante che vende genere alimentari al minuto, salumiere}%
}
