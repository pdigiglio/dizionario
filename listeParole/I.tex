
\newglossaryentry{impermalire}{% 11
    name        = {Impermalire},%
    description = {Stizzire, indispettire, offendere}%
}

\newglossaryentry{ipossia}{% 7
    name        = {Ipossia},%
    description = {Insufficienza d'ossigeno}%
}

\newglossaryentry{ieratico}{% 8
    name        = {Ieratico},%
    description = {\switch{1}~Sacerdotale, per lo più in riferimento a sacerdoti di popoli antichi. \switch{2}~\ex{Scrittura ieratica}, scrittura geroglifica egiziana abitualmente usata dai sacerdoti. \switch{3}~Di cosa improntata ad una compostezza sacerdotale, solenne}%
}

\newglossaryentry{italiota}{% 8
    name        = {Italiota},%
    description = {\switch{1}~Nome con cui gli antichi greci indicavano i coloni della Magna Grecia. \switch{2}~Italiano ottuso e ignorante}%
}

\newglossaryentry{insipienza}{% 10
    name        = {Insipienza},% 2016-05-06
    description = {Ignoranza, stoltezza, ottusità}%
}

\newglossaryentry{ineluttabile}{% 12
    name        = {Ineluttabile},% 2016-05-08
    description = {Contro cui non si può lottare, quindi \emph{inevitabile}}%
}

\newglossaryentry{ingollare}{% 9
    name        = {Ingollare},% 2016-05-10
    description = {Ingoiare, inghittire avidamente}%
}

\newglossaryentry{indolenza}{% 9
    name        = {Indolenza},% 2016-05-21
    description = {\switch{1}~Indifferenza d'animo, inerzia, apatia, pigrizia. \switch{2}~\med~Mancanza di dolore, riferito a parte del corpo malata o ad affezione morbosa che non provochi dolore}%
}

\newglossaryentry{idiosincrasia}{% 13
    name        = {Idiosincrasia},% 2016-05-22
    description = {\switch{1}~\med~Intolleranza di un organismo (non accompagnata né indotta da reazioni immunologiche) verso alcuni agenti esterni quali medicinali o alimenti. \switch{2}~\est~Avversione, ripugnanza, incompatibilità}%
}

\newglossaryentry{imbelle}{% 7
    name        = {Imbelle},% 2016-08-13
    description = {\switch{1}~Non adatto alla guerra. \switch{2}~\est~Fiacco, debole, vile}%
}

\newglossaryentry{icastico}{% 8
    name        = {Icastico},% 2016-08-22
	description = {Che descrive nei tratti essenziali, e quindi in maniera efficace, asciutta e spesso tagliente: \ex{uno stile icastico}}%
}
