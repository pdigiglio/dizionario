\newglossaryentry{caldeggiare}{% 11
	name        = {Caldeggiare},
	description = {Sostenere con fervore}
}

\newglossaryentry{compagine}{% 9
	name        = {Compagine},
    description = {Connessione di varie parti e l'insieme stesso delle parti connesse, anche nel significato di \emph{squadra}}
}

\newglossaryentry{commiato}{% 8
    name        = {Commiato},%
    description = {\switch{1}~Congedo. \switch{2}~Separazione, distacco. \switch{3}~\lett~Parte conclusiva di uno scritto}%
}

\newglossaryentry{contingentare}{% 13
    name        = {Contingentare},%
    description = {Limitare il consumo di un prodotto per ragioni di necessità, razionare}%
}

\newglossaryentry{connivente}{% 10
    name        = {Connivente},%
    description = {Che tacitamente acconsente ad un'azione non buona, pur avendo la possibilità o l'obbligo di opporvicisi}%
}

\newglossaryentry{capobastone}{% 11
    name        = {Capobastone},%
    description = {Nell'organizzazione mafiosa, il capo di un'area territoriale limitata}%
}

\newglossaryentry{contezza}{% 8
    name        = {Contezza},%
    description = {Notizia, cognizione, conoscenza}%
}

\newglossaryentry{correggia}{% 9
    name        = {Correggia},% 2016-05-10
    description = {Cinghia, striscia di cuoio}%
}

\newglossaryentry{coibente}{% 8
    name        = {Coibente},% 2016-05-10
    description = {Riferito a materiale che non lascia passare elettricità, calore o suono; \emph{isolante}}%
}

\newglossaryentry{casba}{% 5
    name        = {Casba},% 2016-05-11
    description = {\switch{1}~Parte interna fortificata delle città arabe. \est~Complesso di quartieri autoctoni di una città araba: \ex{la casba di Algeri}. \switch{2}~\fig~Zona malfamata di una città}%
}

\newglossaryentry{camarilla}{% 9
    name        = {Camarilla},% 2016-05-18
    description = {Dallo spagnolo \emph{camarilla}, l'antisala della camera reale e~\est~i favoriti del re (soprattutto Ferdinando VII di Spagna): gruppo di persone che, senza meriti o riconoscimento ufficiale, occultamente influenzano l'azione di un governo}%
}

\newglossaryentry{canea}{% 5
    name        = {Canea},% 2016-05-18
    description = {\switch{1}~Muta di cani che inseguono la selvaggina abbaiando, e il loro stesso abbaiare. \switch{2}~\est~Folla di persone urlanti. \switch{3}~\fig~Gran clamore di opinioni per un avvenimento pubblico: \ex{la canea dei critici}. \switch{3}~\fig~Schiamazzo, chiassata}%
}

\newglossaryentry{capzioso}{% 8
    name        = {Capzioso},% 2016-05-19
    description = {Che tende a trarre in inganno, fallace, insidioso}%
}

\newglossaryentry{coprolalia}{% 10
    name        = {Coprolalia},% 2016-05-22
	description = {\wom{maggio 2016}{In psichiatria e nel linguaggio letterario, il fatto o l'abitudine di parlare oscenamente; in particolare, impulso anormale a fare riferimento agli escrementi, all'ano ed ai genitali con espressioni volgari}}%
}

\newglossaryentry{coacervo}{% 8
    name        = {Coacervo},% 2016-05-28
    description = {Ammassamento, miscuglio di cose, accozzaglia}%
}

\newglossaryentry{complessione}{% 12
    name        = {Complessione},% 2016-05-31
    description = {\switch{1}~Nell'uomo, costituzione fisica individuale: \ex{``Renzo prese anche lui la peste, \omissis{} ma la sua buona complessione vinse la forza del male''} (Manzoni)}%
}

\newglossaryentry{convolare}{% 9
    name        = {Convolare},% 2016-06-06
    description = {Volare assieme verso una meta, accorrere velocemente. Oggi usato nell'espressione \ex{convolare a nozze} o \ex{convolare a giuste nozze}}%
}

\newglossaryentry{capitolino}{% 10
    name        = {Capitolino},% 2016-06-06
    description = {Del Campidoglio. \est~Del comune di Roma}%
}

\newglossaryentry{coriaceo}{% 8
    name        = {Coriaceo},% 2016-06-25
    description = {\switch{1}~Di cuoio, simile al cuoio. \fig~Persona insensibile, dura, fredda}%
}

\newglossaryentry{coreutico}{% 9
    name        = {Coreutico},% 2016-06-30
    description = {Detto di ciò che attiene all'arte della danza e alla sua fenomenologia}%
}

\newglossaryentry{caustico}{% 8
    name        = {Caustico},% 2016-07-21
	description = {\switch{1}~Che ha la proprietà di cauterizzare tessuti organici. \switch{2}~\fig~Mordace, pungente, aspro: \ex{parole caustiche}}%
}

\newglossaryentry{cauterio}{% 8
    name        = {Cauterio},% 2016-07-21
    description = {\switch{1}~Strumento chirurgico per l'applicazione circoscritta di alte temperature su tessuti organici per causarne la distruzione (cauterizzazione). \switch{2}~\ant~Causticazine circoscritta per favorire l'uscita di <<umori corrotti>> (detta anche \emph{fontanella} o \emph{fonticolo}). \switch{3}~\fig~Angustia morale, pena. \switch{4}~\ant~Persona noiosa, molesta}%
}
\newglossaryentry{circostanziare}{% 14
    name        = {Circostanziare},% 2016-07-08
    description = {Riferire minutamente un fatto in tutte le sue circostanze, nelle sue particolarità: \ex{circostanziare un'accusa}}%
}

\newglossaryentry{ciarpa}{% 6
    name        = {Ciarpa},% 2016-07-24
    description = {(Al plurale) Cosa vecchia, diventata inutile e senza pregio. \fig~(Al singolare) Ciance, fronzoli, parole inutili}%
}

\newglossaryentry{censo}{% 5
    name        = {Censo},% 2016-08-13
    description = {\switch{1}~Patrimonio del cittadino che, per legge, può essere sottoposto a tributo. \switch{2}~\est~Tributo pagato allo Stato in proporzione ai propri averi. \switch{3}~\est~Patrimonio, ricchezza. \switch{4}~\stor~Presso gli antichi Romani, censimento dei cittadini con la registrazione dei beni posseduti da ciascuno}%
}

\newglossaryentry{cassare}{% 7
    name        = {Cassare},% 2016-09-16
    description = {\switch{1}~Cancellare uno scritto in modo che non ne resti più traccia. \switch{2}~Abrogare, revocare; riferito a leggi, decreti, sentenze o disposizioni ufficiali, togliere autorità o effetto}%
}

\newglossaryentry{compassato}{% 10
    name        = {Compassato},% 2016-11-24
    description = {Misurato con cura e precisione, scrupolosamente ponderato: \ex{eloquenza compassata}; indica spesso un affettato controllo}%
}

\newglossaryentry{coartare}{% 8
    name        = {Coartare},% 2017-04-03
    description = {\switch{1}~\lett~Restringere, comprimere: \ex{``soverchia arditezza sarebbe se altri volesse limitare e coartare la divina potenza e sapienza ad una sua fantasia particolare''} (Galilei). \switch{2}~Forzare, costringere moralmente}%
}
