\newglossaryentry{strobilo}{% 8
    name        = {Strobilo},%
    description = {Pigna, cono}%
}

\newglossaryentry{stipo}{% 5
    name        = {Stipo},%
    description = {\switch{1}~Piccolo armadio utilizzato nel Medioevo per conservare preziosi, documenti, \etc. \switch{2}~Armedietto pensile utilizzato nelle cucine e nelle dispense}%
}

\newglossaryentry{salubre}{% 7
    name        = {Salubre},%
    description = {Che giova alla salute (sup.~ass.~\emph{saluberrimo})}%
}

\newglossaryentry{strenna}{% 7
    name        = {Strenna},%
    description = {\switch{1}~Dono offerto o ricevuto in occasioni di festività solenni. \switch{2}~\emph{Libro strenna}, libro in edizione di lusso dedicato alle vendite natalizie. \switch{3}~\stor~Presso gli antichi Romani, dono offerto dapprime al re, poi scambiato tra i cittadini in particolari festività, specialmente per le calende di gennaio}%
}

\newglossaryentry{scranna}{% 7
    name        = {Scranna},%
    description = {\switch{1}~Sedia dottorale, o seggio del giudice, con spalliera e braccioli alti. \ex{Sedere a scranna}, ergersi a giudice. \switch{2}~\reg~In forma rara maschile \emph{scranno}, panca o sedia}%
}

\newglossaryentry{sodalizio}{% 9
    name        = {Sodalizio},%
    description = {\switch{1}~Associazione, società, circolo (specialmente sportivo o artistico): \ex{sodalizio sportivo}. \switch{2}~\lett~Amicizia, familiarità. \switch{3}~\eccl~Nella chiesa cattolica, congregazione, confraternita. \switch{4}~\stor~Nell'antica Roma, associazione politica o religiosa}%
}
