\newglossaryentry{strobilo}{% 8
    name        = {Strobilo},%
    description = {Pigna, cono}%
}

\newglossaryentry{stipo}{% 5
    name        = {Stipo},%
    description = {\switch{1}~Piccolo armadio utilizzato nel Medioevo per conservare preziosi, documenti, \etc. \switch{2}~Armedietto pensile utilizzato nelle cucine e nelle dispense}%
}

\newglossaryentry{salubre}{% 7
    name        = {Salubre},%
    description = {Che giova alla salute (sup.~ass.~\emph{saluberrimo})}%
}

\newglossaryentry{strenna}{% 7
    name        = {Strenna},%
    description = {\switch{1}~Dono offerto o ricevuto in occasioni di festività solenni. \switch{2}~\emph{Libro strenna}, libro in edizione di lusso dedicato alle vendite natalizie. \switch{3}~\stor~Presso gli antichi Romani, dono offerto dapprime al re, poi scambiato tra i cittadini in particolari festività, specialmente per le calende di gennaio}%
}

\newglossaryentry{scranna}{% 7
    name        = {Scranna},%
    description = {\switch{1}~Sedia dottorale, o seggio del giudice, con spalliera e braccioli alti. \ex{Sedere a scranna}, ergersi a giudice. \switch{2}~\reg~In forma rara maschile \emph{scranno}, panca o sedia}%
}

\newglossaryentry{sodalizio}{% 9
    name        = {Sodalizio},%
    description = {\switch{1}~Associazione, società, circolo (specialmente sportivo o artistico): \ex{sodalizio sportivo}. \switch{2}~\lett~Amicizia, familiarità. \switch{3}~\eccl~Nella chiesa cattolica, congregazione, confraternita. \switch{4}~\stor~Nell'antica Roma, associazione politica o religiosa}%
}

\newglossaryentry{surrettizio}{% 11
    name        = {Surrettizio},%
    description = {\switch{1}~In linguaggio giuridico, di atto in cui si tace intenzionalmente un fatto: \ex{esposizione surrettizia dei fatti}. \switch{2}~In filosofia, di concetto introdotto in modo illegittimo poiché in contrasto coi printip\^i fondamentali professati con l'intento non dichiarato di sostenere una particolare tesi}%
}

\newglossaryentry{suburra}{% 7
    name        = {Suburra},% 2016-05-06
    description = {Quartiere più malfamato di una città, dove si concentra la malavita. Dal nome proprio di una zona della Roma antica ceh alla fine della repubblica divenne un quartiere popolare di piccoli commercianti, gente di malaffare e sede di postriboli}%
}

\newglossaryentry{soporoso}{% 8
    name        = {Soporoso},% 2016-05-06
    description = {Che induce sopore, che concilia il sonno}%
}

\newglossaryentry{stillicidio}{% 11
    name        = {Stillicidio},% 2016-05-18
    description = {\switch{1}~Caduta continua e lenta dell'acqua goccia a goccia. \switch{2}~\fig~Il ripetersi continuo, insistente e logorante di qualcosa di fastidioso o dannoso}%
}

\newglossaryentry{scorno}{% 6
    name        = {Scorno},% 2016-05-19
    description = {Grave umiliazione e vergogna, generalmente accompagnata da ridicolo, per un fallimento od insuccesso subito}%
}

\newglossaryentry{sciorare}{% 8
    name        = {Sciorare},% 2016-05-23
    description = {Esporre all'aria, sciorinare, o, con valore intransitivo, spandersi all'aria}%
}

\newglossaryentry{sciorinare}{% 10
    name        = {Sciorinare},% 2016-05-23
    description = {\switch{1}~Stendere, spiegare all'aria (specialmente il bucato). \est~Mettere in mostra. \switch{2}~\fig~Riferire con eccessiva disinvoltura e scarso ritegno cose per lo più riservate; anche esibire, ostentare}%
}

\newglossaryentry{sprimacciare}{% 12
    name        = {Sprimacciare},% 2016-05-31
    description = {Scuotere e sbattere energicamente con le palme delle mani un cuscino, guanciale o altri oggetti imbottiti di piume o lana affinché l'imbottitura si distribuisca con uniformità}%
}

\newglossaryentry{sparuto}{% 7
    name        = {Sparuto},% 2016-05-31
    description = {\switch{1}~Che si mostra smunto, deperito ed emaciato. \switch{2}~\fig~Che è in numero esiguo, quantitativamente irrilevante}%
}

\newglossaryentry{satrapo}{% 7
    name        = {Satrapo},% 2016-06-07
    description = {\switch{1}~Governatore di una provincia dell'antico impero persiano, con amp\^i poteri politici, amministrativi e militari. \switch{2}~\est~Monarca di un paese orientale. \switch{3}~\fig~Persona investita di un certo potere che ostenta e fa pesare la sua autorità, esercita il suo ufficio con sussiego, dandosi un'importanza spropositata alla propria carica. \switch{4}~Persona che vive tra agi e ricchezze. Al femminile, \emph{satrapessa} indica una donna troppo autoritaria piuttosto che la moglie del satrapo}%
}

\newglossaryentry{sicofante}{% 9
    name        = {Sicofante},% 2016-06-16
	description = {\switch{1}~Nel diritto attico, e in quello di altre città a regime democratico dell'antica Grecia, persona che denunciava violazioni della legge alle autorità di propria iniziativa. \switch{2}~\est~Delatore, calunniatore, spia}%
}

\newglossaryentry{sciuscià}{% 8
    name        = {Sciuscià},% 2016-06-27
    description = {\upag{(dall'inglese \emph{shoe-shine}, lustrascarpe)~--- Ragazzo che nel secondo dopoguerra faceva lavori per gli alleati; piccolo criminale}}%
}
