
\newglossaryentry{magione}{% 7
    name        = {Magione},%
    description = {\lett~Palazzo, casa, abitazione}%
}

\newglossaryentry{millanteria}{% 11
    name        = {Millanteria},%
    description = {Eccessivo sentimento di sé, boria, vanteria}%
}

\newglossaryentry{marasma}{% 7
    name        = {Marasma},%
    description = {\switch{1}~\med~Stato di deperimento grave ma ancora reversibile caratterizzato da estrema magrezza. \switch{2}~\fig~Stato di grave disordine e decadenza di istituzioni sociali, politiche \etc}%
}

\newglossaryentry{meritorio}{% 9
    name        = {Meritorio},% 2016-05-13
	description = {Che costituisce o procura merito, degno di lode o ricompensa morale (anche ironico: \ex{compiresti opera meritoria a lasciarmi in pace}). Diverso da \emph{meritevole}, generalmente è riferito alla persona che compie l'azione o in espressioni come \ex{azione meritevole d'attenzione} (che, quindi, non procura merito ma ne dovrebbe ricevere)}%
}

\newglossaryentry{memento}{% 7
    name        = {Memento},% 2016-05-13
    description = {\switch{1}~\eccl~Ciascuna delle due preghiere in cui si commemorano i morti e i vivi. \switch{2}~\est~Ammonimento, lezione}%
}

\newglossaryentry{minutante}{% 9
    name        = {Minutante},% 2016-05-13
    description = {Chi ha negozio di vendita a minuto, in contrapposizione a \emph{grossista}. Più comunemente \emph{dettagliante}}%
}

\newglossaryentry{moratoria}{% 9
    name        = {Moratoria},% 2016-05-18
    description = {Sospensione della scadenza delle obbligazioni in genere (e specialmente pecuniarie) disposta con provvedimento legislativo in via eccezionale e con riferimento a eventi straordinari tali da turbare il normale svolgimento dei rapporti economici e sociali}%
}

\newglossaryentry{minculpop}{% 9
    name        = {Minculpop},% 2016-06-13
    description = {(per \emph{Ministero della Cultura Popolare}) --- Nell'Italia fascista, denominazione abbreviata del ministero per il controllo degli affari culturali, della stampa e delle pubblicazioni in genere; oggi il termine è usato in tono spregiativo}%
}

\newglossaryentry{moria}{% 5
    name        = {Moria},% 2016-09-03
    description = {\switch{1}~Alta mortalità di uomini o animali per malattie epidemiche. \switch{2}~In botanica, il deperimento e la successiva morte di alcune specie di piante in seguito ad infezioni parassitarie}%
}

\newglossaryentry{morigerato}{% 10
    name        = {Morigerato},% 2016-09-19
    description = {Di sani e regolati costumi, sobrio, misurato: \ex{condurre una vita morigerata}}%
}

\newglossaryentry{meriggio}{% 8
    name        = {Meriggio},% 2016-10-04
    description = {\wom{ottobre 2016}{\switch{1}~Le ore attorno a mezzogiorno, in cui è più intenso il caldo e il sole è più alto sull'orizzonte. \switch{2}~\fig~Il culmine di un periodo storico, culturale, \etc. \switch{3}~Il mezzogiorno come punto cardinale, il sud: \ex{esposto a meriggio}. \switch{4}~\pop~Luogo ombreggiato dove si trascorrono in riposo le ore del meriggio. \est~Il riposo stesso: \ex{far meriggio}, \ex{meriggiare}}}%
}

\newglossaryentry{maretta}{% 7
    name        = {Maretta},% 2017-02-12
    description = {\switch{1}~Stato di agitazione della superficie marina provocato dal vento, caratterizzato da onde poco alte e spumeggianti al vertice. \switch{2}~\fig~Tensione, agitazione e malcontento celati o parzialmente repressi}%
}

\newglossaryentry{mucido}{% 6
    name        = {Mucido},% 2017-06-13
    description = {Ciò che odora o sa di muffa per essere stato in un luogo umido; ammuffito, stantio}%
}

\newglossaryentry{melomane}{% 8
    name        = {Melomane},% 2017-11-12
    description = {Chi ha una passione smodata per la musica (specialmente lirica)}%
}
