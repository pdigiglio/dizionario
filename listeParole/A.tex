\newglossaryentry{accomiatare}{% 11
	name        = {Accomiatarsi},
	description = {Dare commiato, congedare},
}

\newglossaryentry{ascesso}{% 7
	name        = {Ascesso},%
	description = {Raccolta di pus in una cavità determinata del corpo a causa di un'infiammazione virulenta},%
}

\newglossaryentry{alabastrino}{% 11
    name        = {Alabastrino},%
    description = {D'alabastro}%
}

\newglossaryentry{acquiescente}{% 12
    name        = {Acquiescente},%
    description = {Che acconsente con remissività, accondiscendente, arrendevole}%
}

\newglossaryentry{afferente}{% 9
    name        = {Afferente},%
    description = {\switch{1}~\anat~Di organo che ha una funzinone confuttiva verso una parte del corpo: \ex{vasi afferenti}, che portano il sangue o la linfa ad un determinato organo. \switch{2}~Riguardante qualcosa}%
}

\newglossaryentry{afflato}{% 7
    name        = {Afflato},%
    description = {\lett~Alito, soffio, soprattuto in senso figurato: \ex{afflato poetico}}%
}

\newglossaryentry{alterco}{% 7
    name        = {Alterco},%
    description = {Litigio, discussione molto animata}%
}

\newglossaryentry{arnia}{% 5
    name        = {Arnia},% 2016-05-02
    description = {Alveare, cassetta per l'allevamento delle api}%
}

\newglossaryentry{acribia}{% 7
    name        = {Acribia},% 2016-05-06
    description = {Esattezza, meticolosa precisione}%
}

\newglossaryentry{auscultazione}{% 13
    name        = {Auscultazione},% 2016-05-11
    description = {\med~Esame degli organi interni di un malato consistente nell'ascoltare i suoni da essi provenienti per valutarne eventuali anomalie}%
}

\newglossaryentry{acefalia}{% 8
    name        = {Acefalia},% 2016-05-12
    description = {L'essere privo di capo}%
}

\newglossaryentry{agro}{% 4
    name        = {Agro},% 2016-05-13
	description = {Campagna attorno ad una città; spesso usato come nome di alcuni antichi insediamenti con proprie caratteristiche agrarie: \ex{Agro Romano}, \ex{Agro Pontino}. Nell’età romana, l’\foreignlanguage{latin}{\emph{ager}} era il territorio di uno stato politicamente costituito: così \foreignlanguage{latin}{\emph{ager Romanus}} indicò il territorio di Roma nel periodo regio e anche, talora, tutto il territorio romano; \foreignlanguage{latin}{\emph{ager peregrinus}} il territorio di stati riconosciuti da Roma; \foreignlanguage{latin}{\emph{ager hosticus}} il territorio confiscato ai vinti}%
}

\newglossaryentry{agone}{% 5
    name        = {Agone},% 2016-05-19
    description = {\switch{1}~Ciascuna delle gare, ginniche, ippiche e musicali, che si svolgevano tra gli antichi greci per la conquista di premi: \ex{gli agoni olimpici}. \switch{2}~\lett~Lotta, gara e, per metonimia, il campo stesso di gara}%
}
