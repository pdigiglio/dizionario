
\newglossaryentry{rebbio}{% 6
    name        = {Rebbio},% 2016-05-10
    description = {\switch{1}~Ciascuna delle punte di forca, forcone, tridente, forchetta o simili. \switch{2}~\est~Ciascuno dei bracci del diapason. \switch{3}~\est~Ciascuno dei bracci delle manette per la traduzione dei detenuti}%
}

\newglossaryentry{riottoso}{% 8
    name        = {Riottoso},% 2016-05-21
    description = {Litigioso e insofferente di ogni norma e disciplina, ribelle alle imposizioni e ai consigli}%
}

\newglossaryentry{rutilante}{% 9
    name        = {Rutilante},% 2016-06-05
	description = {\wom{giugno 2016}{Rosso vivo; più genericamente, risplendente}}%
}

\newglossaryentry{reticenza}{% 9
    name        = {Reticenza},% 2016-06-09
    description = {\switch{1}~Il tacere volontariamente qualcosa che si dovrebbe o potrebbe dire: \ex{la reticenza del testimone costituisce reato}. \switch{2}~Figura retorica (anche detta, con parola greca, \emph{aposiopesi}) che consiste in una sospensione improvvisa della frase lasciando tuttavia capire quanto si è omesso}%
}

\newglossaryentry{redivivo}{% 8
    name        = {Redivivo},% 2016-10-05
    description = {\switch{1}~Ritornato in vita, rinato a nuova vita: \ex{tornerà il giorno in cui redivivi omai gl’Itali staranno in campo audaci} (V.~Alfieri). \switch{2}~\iron~Che ridà sue notizie dopo lunga assenza o silenzio. \switch{3}~\iperb~Per indicare somiglianza con un personaggio del passato: \ex{è suo padre redivivo}}%
}

\newglossaryentry{risma}{% 5
    name        = {Risma},% 2016-10-24
    description = {\switch{1}~Unità di conto per la carta equivalente a $500$ fogli per la carta da stampa e $400$ per la carta da cancelleria. \switch{2}~\fig~Genere, razza, specie di persone, sempre in senso spregiativo: \ex{un furfante della peggior risma}}%
}

\newglossaryentry{rampogna}{% 8
    name        = {Rampogna},% 2016-10-26
    description = {Rimprovero severo con cui si biasima il comportamento di qualcuno}%
}
