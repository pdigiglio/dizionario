
\newglossaryentry{libagione}{% 9
    name        = {Libagione},%
    description = {\switch{1}~Offerta sacrificale di sostenze liquide (vino, miele, latte, \etc). \switch{2}~Bevuta (abbondante) di vino e alcolici}%
}

\newglossaryentry{lasco}{% 5
    name        = {Lasco},%
    description = {Largo, allentato, rilassato: \ex{una morale lasca}}%
}

\newglossaryentry{lapidario}{% 9
    name        = {Lapidario},% 2016-05-02
    description = {\switch{1}~Che si riferisce alle iscrizioni su lapide. \switch{2}~\fig~Sentenzioso, grave e conciso}%
}

\newglossaryentry{lezio}{% 5
    name        = {Lezio},% 2016-05-09
    description = {Atto affettato e svenevole, smanceria, moina: \ex{non posso soffrire i suoi lez\^i}}%
}

\newglossaryentry{laterizio}{% 9
    name        = {Laterizio},% 2016-05-10
	description = {\switch{1}~\edil~Di terracotta, di mattoni. \switch{2}~Denominazione generica di materiali artificiali da costruzione d'argilla: \ex{lateriz\^i pieni} sono comunemente detti \emph{mattoni} (per murature portanti)}%
}

\newglossaryentry{livore}{% 6
    name        = {Livore},% 2016-05-11
    description = {\switch{1}~Astio maligno, rabbia velenosa, rancore. \switch{2}~\ant~Aspetto livido}%
}

\newglossaryentry{lambire}{% 7
    name        = {Lambire},% 2016-05-13
    description = {\lett~Leccare delicatamente. \est~Toccare appena, sfiorare,  riferito soprattutto ad acque correnti o in movimento, fuoco \etc{}}%
}

\newglossaryentry{laticlavio}{% 10
    name        = {Laticlavio},% 2016-08-24
    description = {Nell'antica Roma, la tunica orlata di una stricia di porpora indossata dai senatori e poi dai membri maggiorenni delle famiglie senatoriali. \est~La dignità di senatore: \ex{ottenere il laticlavio}}%
}
