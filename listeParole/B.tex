
\newglossaryentry{bisboccia}{% 9
    name        = {Bisboccia},%
    description = {Mangiata e bevuta allegra tra amici, baldoria, gozzoviglia}%
}

\newglossaryentry{bega}{% 4
    name        = {Bega},%
    description = {\switch{1}~Litigio, contrasto (per cose futili). \switch{2}~Impiccio, seccatura, grattacapo}%
}

\newglossaryentry{bigio}{% 5
    name        = {Bigio},%
    description = {\switch{1}~Di colore grigio cenere. \switch{2}~Indeciso, specialmente di chi in politica non sa decidere per uno o l'altro partito}%
}

\newglossaryentry{belluino}{% 8
    name        = {Belluino},% 2016-05-11
    description = {Che ha natura o aspetto di belva, bestiale. \fig~Feroce, inumano}%
}

\newglossaryentry{becero}{% 6
    name        = {Becero},% 2016-05-13
    description = {Che manifesta la sua rozzezza d'animo e di abitudini con modi volgari ed insolenti}%
}

\newglossaryentry{bilioso}{% 7
    name        = {Bilioso},% 2016-05-21
    description = {Di bile, pieno di bile. \est~(Riallacciandosi alla dottrina umorale ippocratica) irritabile, collerico}%
}

\newglossaryentry{bugigattolo}{% 11
    name        = {Bugigattolo},% 2016-05-23
    description = {Stanzino piccolo e buio usato per lo più come ripostiglio e, in generale, locale assai piccolo}%
}

\newglossaryentry{bailamme}{% 8
    name        = {Bailamme},% 2016-06-03
    description = {(dal turco \emph{bayram}, festa) --- Confusione e grida di gente cha va e viene, baraonda}%
}

\newglossaryentry{bieco}{% 5
    name        = {Bieco},% 2016-06-07
    description = {\switch{1}~Obliquo, detto dello sguardo, specialmente di chi guarda con malanimo, astio o minacciosità. \switch{2}~\est~Losco, malvagio, che medita od è volto al male: \ex{azioni bieche}; minaccioso o sinistro}%
}

\newglossaryentry{blandire}{% 8
    name        = {Blandire},% 2016-07-19
    description = {Lusingare, allettare con parole carezzevoli. Trattare con modi blandi, rivolgere parole buone: \ex{``soccorre i piccoli infermi, li conforta, li blandisce''} (D’Annunzio). Lenire, mitigare: \ex{blandire il dolore}}%
}
