
\newglossaryentry{emaciare}{% 8
    name        = {Emaciare},%
    description = {Far diventare magro e (trans.) diventare magro}%
}

\newglossaryentry{epiteto}{% 7
    name        = {Epiteto},%
    description = {\switch{1}~Sostantivo o aggettivo che si aggiunge ad un nome per meglio qualificarlo. \switch{2}~Attributo, denominazione, titolo. \switch{3}~Titolo ingiurioso}%
}

\newglossaryentry{egida}{% 5
    name        = {Egida},%
    description = {\switch{1}~\mit~Scudo di Zeus o capo indossato da Atena. Si ritiene fatto di pelle di capra ma da un'opera di Eschilo s'ipotizza possa essere un cerchio di nubi che si addensa attorno alla testa di Zeus al momento del tuono divino. In alcune versioni dello scudo magico di Atena compare la testa della Gorgone. \switch{2}~Protezione, difesa}%
}

\newglossaryentry{emerito}{% 7
    name        = {Emerito},%
    description = {\switch{1}~\stor~Nella Roma antica il soldato (\emph{\foreignlanguage{latin}{emeritus miles}}) che aveva compiuto il servizio militare e ricevuto congedo e relativi premi. \switch{2}~Che non esercita più il suo ufficio ma ne conserva gli onori e talvolta l'emolumento. \switch{3}~Usato come egregio, insigne per accostamento con \emph{benemerito}}%
}

\newglossaryentry{endemico}{% 8
    name        = {Endemico},%
    description = {Proprio di un determinato territorio, detto di malattie}%
}

\newglossaryentry{edotto}{% 6
    name        = {Edotto},% 2016-05-05
	description = {\switch{1}~\emph{Ed\`otto:} informato di qualcosa (\ex{sono già ed\`otto dei fatti}). \switch{2}~\emph{Ed\'otto:} tratto o condotto fuori}%
}

\newglossaryentry{edulcorare}{% 10
    name        = {Edulcorare},% 2016-05-06
    description = {Sinonimo non comune di \emph{dolcificare}. \fig~Mittigare, attenuare nella sua gravità}%
}

\newglossaryentry{estemporaneo}{% 12
    name        = {Estemporaneo},% 2016-06-06
    description = {Fatto, pronunciato o scritto lì per lì, senza preparazione; improvvisato}%
}

\newglossaryentry{esiziale}{% 8
    name        = {Esiziale},% 2016-06-13
    description = {Rovinioso: \ex{un errore esiziale}; che reca grave danno, mortale: \ex{quel clima è esiziale per gli europei}}%
}

\newglossaryentry{esizio}{% 6
    name        = {Esizio},% 2016-06-13
    description = {(dal latino \foreignlanguage{latin}{\emph{exitum}}: l'uscita, l'andare a finire) --- Rovina, eccidio}%
}

\newglossaryentry{esautorare}{% 10
    name        = {Esautorare},% 2016-07-24
    description = {Privare dell'autorità una persona o un organismo cui sono attribuite funzioni direttive o di comando. \est~Privare qualcuno o qualche istituto della stima, del credito, della reputazione rendendo impossibile o difficoltoso l'esercizio delle sue funzioni}%
}

\newglossaryentry{espungere}{% 9
    name        = {Espungere},% 2016-08-13
	description = {\wom{agosto 2016}{\switch{1}~In manoscritti antichi, eliminare dal testo, per correzione, parole, frasi o lettere di una parola, segnando un punto sopra o sotto le singole lettere. \switch{2}~Nella critica testuale, eliminare dal testo parole o frasi presenti nella tradizione del testo ma ritenute interpolate. \switch{3}~\est~Sopprimere parole o passi da un testo}}%
}

\newglossaryentry{erubescente}{% 11
    name        = {Erubescente},% 2016-08-13
    description = {\lett~Che diventa rosso; anche di persona che arrossisce per vergogna o pudore}%
}

\newglossaryentry{eponimo}{% 7
    name        = {Eponimo},% 2016-09-04
    description = {Che dà il proprio nome a qualcosa: \ex{Atena è la dea eponima di Atene}}%
}

\newglossaryentry{eccepire}{% 8
    name        = {Eccepire},% 2016-09-28
    description = {Fare eccezione, opporre obiezione, addurre in contrario}%
}
