
\newglossaryentry{diafano}{% 7
    name        = {Diafano},%
	description = {\switch{1}~Trasparente, terso, limpido: \ex{un'atmosfera diafana}. \switch{2}~Tenue, pallido: \ex{la luce diafana dell'alba}. \switch{3}~Esile, di aspetto delicato e colorito pallido}%
}

\newglossaryentry{diacono}{% 7
    name        = {Diacono},%
    description = {\switch{1}~\eccl~Nella Chiesa cattolica, chi ha ricevuto il diaconato (ordine permanente immediatamente inferiore al prete). \switch{2}~\stor~Prima del Concilio Ecumenico Vaticano II, sacerdote che in alcune funzioni solenni assisteva il celebrante}%
}

\newglossaryentry{daspo}{% 5
    name        = {Daspo},%
    description = {Acronimo di Divieto di Accedere a manifestazioni Sportive}%
}

\newglossaryentry{dacia}{% 5
    name        = {Dacia},% 2016-05-03
    description = {(dal russo \emph{da\v{c}a}) --- Casa di campagna, villetta circondata da giardino, in Russia}%
}

\newglossaryentry{dirimere}{% 8
    name        = {Dirimere},% 2016-05-07
    description = {\switch{1}~Dividere, separare: \ex{``dirimendo del fior tutte le chiome''} (Dante). \switch{2}~Risolvere e troncare, solitamente con l'intervento di un'autorità: \ex{dirimere una controversia}}%
}

\newglossaryentry{divellere}{% 9
    name        = {Divellere},% 2016-05-21
    description = {Strappare con forza, estirpare}%
}

\newglossaryentry{drappello}{% 9
    name        = {Drappello},% 2016-06-03
    description = {\switch{1}~Piccolo drappo foggiato a banderuola come insegna di corpi militari. \switch{2}~Un numero di soldati raccolti sotto la medesima insegna non raggruppati in un reparto organico. \est~Schiera di persone raccolte assieme}%
}

\newglossaryentry{detrattore}{% 10
    name        = {Detrattore},% 2016-06-10
    description = {Chi cerca di nuocere alla reputazione di qualcuno con la maldicenza o critiche maligne}%
}

\newglossaryentry{delatore}{% 8
    name        = {Delatore},% 2016-06-16
    description = {Colui che per denaro o interesse denuncia all'autorità un reato e il suo autore}%
}

\newglossaryentry{droghiere}{% 9
    name        = {Droghiere},% 2016-06-29
    description = {(derivato di \emph{droga})~--- Negoziante che vende al minuto spezie e altri generi coloniali (detti, in passato, \emph{droghe})}%
}

\newglossaryentry{deplorare}{% 9
    name        = {Deplorare},% 2016-09-02
    description = {\switch{1}~Riprovare, condannare con biasimo e profondo rammarico: \ex{deploro la vostra condotta}. \switch{2}~\ant~Compiangere, compatire: \ex{deplorare le disgrazie altrui}. \emph{Deplorevole} può essere usato in questo significato, spesso indicando una sfumatura di condanna: \ex{le strade sono in condizioni deplorevoli}}%
}

\newglossaryentry{distinta}{% 8
    name        = {Distinta},% 2016-11-16
    description = {Nota o elenco contenente i dati specifici relativi ad oggetti e simili: \ex{distinta di negoziazione}, documento bancario relativo alla negoziazione di divise estere}%
}

\newglossaryentry{derubricare}{% 11
    name        = {Derubricare},% 2016-11-24
    description = {Nel ling.~forense e giornalistico, assegnare (da parte di un giudice) ad un reato una qualificazione diversa e meno grave di quella indicata nella \emph{rubrica di reato}}%
}
