
\newglossaryentry{orpello}{% 7
    name        = {Orpello},%
    description = {\switch{1}~Lega di rame, zinco e stagno, di aspetto simile all'oro di solito ridotta in sottili lamine e usata come ornamento. \switch{2}~\pl~Ornamenti superflui e di cattivo gusto. \switch{2}~\fig~Apparenza ingannevole: \ex{la loro generosità è un orpello}}%
}

\newglossaryentry{ostensibile}{% 11
    name        = {Ostensibile},% 2016-06-17
    description = {Che può essere mostrato: \ex{il documento è stato depositato nell'ufficio e ostensibile a chiunque fosse interessato}}%
}

\newglossaryentry{opulenza}{% 8
    name        = {Opulenza},% 2016-06-17
    description = {Ricchezza, grande abbondanza, copiosità, dovizia}%
}

\newglossaryentry{orbo}{% 4
    name        = {Orbo},% 2016-06-24
    description = {\switch{1}~Cieco, privo di vista. \fig~Privo di senno, di giudizio, di discernimento: \ex{``vecchia fama nel mondo li chiama orbi''} (Dante). \switch{2}~\lett~Privo: \ex{``stette la spoglia immemore orba di tanto spiro''} (Manzoni), soprattutto privato di persona cara per opera della morte: \ex{orbo della moglie}}%
}

\newglossaryentry{obolo}{% 5
    name        = {Obolo},% 2016-07-01
    description = {\switch{1}~Nella Grecia antica, sesta parte della dracma. \switch{2}~In età romana, la moneta spicciola in genere. \switch{3}~\ex{Obolo di Caronte}, la moneta che, secondo la credenza popolare degli antichi Greci, ogni morto doveva pagare a Carone perché lo traghettasse di là dall'Acheronte e che, perciò, veniva posta dai parenti in bocca al defunto. \switch{4}~\est~Offerta in denaro di scarsa entità}%
}

\newglossaryentry{oberato}{% 7
    name        = {Oberato},% 2016-06-29
    description = {\switch{1}~Presso gli antichi romani, detto di debitore che, non potendo rimettere i suoi debiti verso un creditore, ne diveniva schiavo. \switch{2}~\fig~Sovraccarico, eccessivamente gravato}%
}
